\documentclass{amsart}

\usepackage[utf8x]{inputenc}
\usepackage{graphicx}
\usepackage{caption}
%\usepackage{epstopdf}
\usepackage{enumerate}
\usepackage{amsmath,amsfonts,amsthm,amssymb}
\usepackage{mathrsfs}
\usepackage{url}
\usepackage{acronym}
\usepackage{thmtools}
\usepackage{nicefrac}
\usepackage{pseudocode}
% \usepackage[authoryear]{natbib}
\usepackage{wrapfig}

% Clever references
% \usepackage{cleveref}
% 
% \crefname{equation}{}{}
% \Crefname{equation}{}{}
% \crefname{figure}{figure}{figures}
% \Crefname{figure}{Figure}{Figures}
% \crefname{appendix}{Appendix}{Appendices}
% \Crefname{appendix}{Appendix}{Appendices}
% \crefname{section}{Section}{Sections}
% \Crefname{section}{Section}{Sections}

% Own stuff
\usepackage{../mathdefs}
\usepackage{../notecommand}
\usepackage{../mytheorems}
\usepackage{../other}

\usepackage[pdftex,unicode,colorlinks=true,linkcolor=black,citecolor=black,hypertexnames=true]{hyperref}
\hypersetup{pdfauthor={},pdftitle={Common representation of acquisition geometries in tomography}}

% Figure are placed in figures directory.
\graphicspath{{Pictures/}}

% Acronyms (depends on the acronym package)
\acrodef{em}[EM]{electron microscopy}
\acrodef{EM}[EM]{Electron Microscopy}
\acrodef{et}[ET]{electron tomography}
\acrodef{ET}[ET]{Electron Tomography}
\acrodef{stem}[STEM]{scanning transmission electron microscopy}
\acrodef{psf}[PSF]{point spread function}
\acrodef{PSF}[PSF]{Point Spread Function}
\acrodef{ctf}[CTF]{contrast transfer function}
\acrodef{CTF}[CTF]{Contrast Transfer Function}
\acrodef{nufFt}[FT]{non-uniform fast Fourier transform}
\acrodef{NufFt}[FT]{Non-uniform fast Fourier transform}
\acrodef{NUFFT}[FT]{Non-Uniform Fast Fourier Transform}
\acrodef{Ft}[FT]{Fourier transform}
\acrodef{FT}[FT]{Fourier Transform}
\acrodef{snr}[SNR]{signal-to-noise ratio}
\acrodef{SNR}[SNR]{Signal-to-Noise Ratio}


\title{Common representation of acquisition geometries in tomography}
\author{}
\date{\today}

% \usepackage{other}

\newcommand*{\Dinv}{{\ensuremath{D^{-1}}}}
\renewcommand*{\phi}{\varphi}

% \setlength{\parindent}{0pt}

\begin{document}

\maketitle




\section{General theory}
\label{sec:general}
%
%
%
During data acquisition, detector and sample move relative to each other. Thus, the position of a specific point on the detector varies 
with time or some other parameter, like a rotation angle. Furthermore, we consider detectors which are surface-, curve- or point-like and 
can be regarded as manifolds in $\RR^D$ parameterized over an open set in $\RR^d$ with typically $d < D$. Hence, we use the following 
representation:\\
Let $T \subset \RR^p$ be a set of (time, angle, \ldots) parameters and $U \subset \RR^d$ a set of detector parameters, e.g. $x$ and $y$ 
coordinates on a flat 2D detector. We call a mapping
%
\begin{equation}
 \label{eq:general:detector}
 X: T \times U \longrightarrow \RR^D
\end{equation}
%
the \emph{detector trace parametrization} and $X(T \times U) \subset \RR^D$ the \emph{detector trace}. For fixed $t \in T$, we define the 
\emph{detector parametrization}
%
\begin{equation}
 \label{eq:general:detector_fixedfirst}
 X_t = X(t, \cdot) : U \Lto \RR^D
\end{equation} 
%
and call $X_t(U) \subset \RR^D$ the detector surface (at time $t$).\\[1ex]
%
%
To model the very general situation of directional input (like rays) to the detector, we further define the \emph{directional field}
%
\begin{equation}
 \label{eq:general:dirfield}
 N: T \times U \Lto \SPHERE^{D-1}.
\end{equation}
%
For $t \in T$ and $u \in U$, the value $N(t, u) \in \SPHERE^{D-1}$ stands for the orientation of the detector at the point $X(t,u)$ caused 
by e.g. collimators. Often, $N(t, u)$ is the unit normal to the detector surface at $X(t, u)$. \\[1ex]
%
%
In the case that one detector point ``sees'' more than one incoming direction, we additionally define the 
\emph{detector pupil}
%
\begin{equation}
 \label{eq:general:detector_pupil}
 P: T \times U \Lto \mathrm{P}(\SPHERE^{D-1})
\end{equation} 
%
which maps $t, u$ to the subset $P(t, u) \subset \SPHERE^{D-1}$ of directions seen by the detector point at $X(t, u)$. This covers, for 
example, the case of PET where the detector pixels register not only perpendicular photons but also photons coming in at an angle.\\[1ex]
%
%
Finally, if the incoming ``radiation'' is not travelling along straight lines, we need to provide this information, too. However, we will 
restrict ourselves to the case where we can uniquely trace such a ray back from a detector point through the sample. We thus define a 
mapping
%
\begin{equation}
 \label{eq:general:rays}
 \gamma: T \times U \times \SPHERE^{D-1} \Lto C_{\mathrm{pw}}^1\big([0,\infty), \RR^d\big)
\end{equation} 
%
where for $t \in T$, $u \in U$ and $\theta \in \SPHERE^{D-1}$, the function $\gamma(t, u, \theta)$ represents the ray arriving at the 
detector point $X(t, u)$ from the direction $\theta$. The subscript ``pw'' stands for ``piecewise'' with the restriction that the lengths 
of such pieces must be bounded from below by a positive constant.  If $P(t,u) = \lbrace N(t,u)\rbrace$, we write 
%
\begin{equation}
 \label{eq:general:ray_onedir}
 \tilde \gamma: T \times U \Lto C_{\mathrm{pw}}^1\big([0,\infty), \RR^d\big),\quad \tilde \gamma(t,u) = \gamma\big(t, u, N(t,u)\big).
\end{equation}
%
%
Let now $\Omega \subset \RR^D$ be a suitable set and $\big(\SPCH_j, \INNER{\cdot}{\cdot}_j\big)$, $j=1,2$ be Hilbert spaces. In this 
setting, we can study forward operators of the form
%
\begin{align}
 & \OPA: \SPCH_1(\Omega) \Lto \SPCH_2(T \times U) \notag \\
 \label{eq:general:fwdproj}
 & \OPA f(t, u) = \int_{P(t, u)} \int_{\gamma(t, u, \theta)} f \, \D \gamma\, \D \theta.
\end{align}
%
which maps a pair of parameters $(t, u)$ to the integral of $f$ along all rays arriving at $X(t, u)$. For $L^2$ spaces, we want to 
calculate the adjoint operator:
%
\begin{align}
 \INNER{\OPA f}{g}_2 
 &= \int_T \int_U \OPA f(t, u)\, g(t, u)\, \D u\, \D t \notag \\
 &= \int_T \int_U \int_{P(t, u)} \int_{\gamma(t, u, \theta)} f \, \D \gamma\, \D \theta\, g(t, u)\, \D u\, \D t \notag \\
 \label{eq:general:backproj_calculation_step1}
 &= \int_T \int_U \int_{P(t, u)} \int_0^\infty  [f \circ \gamma(t, u, \theta)](s)\, \ABS{\gamma(t, u, \theta)'(s)}\, g(t, u)\, \D s\, 
 \D \theta\, \D u\, \D t.
\end{align}
%
We first consider the case where $P(t,u) = \lbrace N(t,u)\rbrace$, i.e. there is exactly one ray arriving at each detector point 
$X_t(u)$ with incoming direction $N_t(u)$. For fixed $t \in T$, we assume that for each $x \in \Omega$, there is a unique ray 
$\tilde \gamma_t(u) = \tilde \gamma_t(u; x)$ containing the point $x$ exactly once, i.e. $\tilde \gamma_t(u; x)(s(x)) = x$. Thus, we can 
define a 
mapping
%
\begin{equation*}
 \Gamma_t: \Omega \to [0, \infty) \times U,\quad x \mapsto (s, u) \text{ with } \tilde \gamma_t(u)(s) = x,
\end{equation*}
%
which is the inverse of the mapping $(s,u) \mapsto \tilde \gamma_t(u)(s)$. The point $X_t\big(u(x)\big)$ can be interpreted as the 
projection of 
$x$ to the detector surface along the corresponding ray, and $s(x)$ is the arc length along the ray from $x$ to its projection. 
Now we can rewrite the integral \eqref{eq:general:backproj_calculation_step1} as
%
\begin{align*}
 \INNER{\OPA f}{g}_2
 &= \int_T \int_{\Gamma_t(\Omega)} f\big(\Gamma_t^{-1}(s,u)\big) \ABSLR{\partial_s \Gamma_t^{-1}(s,u)}\, g(t, u)\, \D s\, \D u\, 
 \D t  \\
 &= \int_T \int_{\Gamma_t(\Omega)} f\big(\Gamma_t^{-1}(s,u)\big) \ABSLR{\big[\partial \Gamma_t\big(\Gamma_t^{-1}(s,u)\big)\big]_1}^{-1}\, 
 g(t, u)\, \D  s\, \D u\, \D t \\
 &= \int_T \int_\Omega f(x) \ABSLR{\big[\partial \Gamma_t(x)\big]_1}^{-1}\, \ABS{\DET{\partial \Gamma_t(x)}}\, 
 g\big(t, \Pi_{\gamma_t}(x)\big)
 \, \D x\, \D t,
\end{align*}
%
where $[\partial \Gamma_t]_1$ stands for the first column of the Jacobian of $\Gamma_t$ and $\Pi_{\gamma_t}(x) = [\Gamma_t(x)]_2$ is the 
$u$ component of $(s, u) = \Gamma_t(x)$. Note that $\ABS{[\partial \Gamma_t]_1} = 1$ if the rays are parametrized with respect to arc 
length. Hence, the adjoint operator can be written as
%
\begin{equation}
 \label{eq:general:backproj_onedir}
 \DUALOPA g(x) = \int_T \ABSLR{\big[\partial \Gamma_t(x)\big]_1}^{-1}\, \ABS{\DET{\partial \Gamma_t(x)}}\, g\big(t, 
\Pi_{\gamma_t}(x)\big)\, 
 \D t
\end{equation}
%
which is equal to
%
\begin{equation}
 \label{eq:general:backproj_onedir_alt}
 \DUALOPA g(x) = \int_T \ABSLR{\partial_s \Gamma_t^{-1}\big(\Gamma_t(x)\big)}\, 
 \ABS{\DET{\partial \Gamma_t^{-1}\big(\Gamma_t(x)\big)}}^{-1}\, g\big(t, \Pi_{\gamma_t}(x)\big)\, \D t.
\end{equation}
%
If the transformation factors cannot be determined, it is still possible to calculate the adjoint
%
\begin{align}
 \OPA^\#: \SPCH_2(T \times U, w) \Lto \SPCH_1(\Omega) \notag \\
 \label{eq:general:backproj_onedir_weighted}
 \OPA^\# g(x) = \int_T g\big(t, \Pi_{\gamma_t}(x)\big)\, \D t
\end{align}
%
on the weighted image space $\SPCH_2$ with unknown weight $w: T \times U \to [0, \infty)$.%
\NOTE{TODO: consider the second case, too}
\vspace{5ex}%





\section{Applications}
\label{sec:applications}


\subsection{Parallel geometry}
\label{sec:applications:parbeam}

We consider functions on $\Omega= B_1 \subset \RR^3$ and a flat 2D detector moving on the unit circle in the $x$-$y$ 
plane, oriented to the center of the circle:
%
\begin{align*}
 & T = [-\phi_0, \phi_0],\quad U = [-l_y, l_y] \times [-l_z, l_z] \\
 & X(\phi, u) = \theta(\phi) + \rho(\phi) \cdot \TRANSP{(0, u_1, u_2)}
\end{align*}
%
with $\phi_0 \in (0, \pi/2]$, $l_y, l_z > 0$ and
%
\begin{align}
 \theta(\phi) &= \TRANSP{(\cos\phi, \sin\phi, 0)}, \\ 
 \rho(\phi) &=
 \begin{pmatrix}
  \cos\phi & -\sin\phi & 0 \\
  \sin\phi & \cos\phi & 0 \\
  0 & 0 & 1
 \end{pmatrix}
 = \big(\theta(\phi) | \theta'(\phi) | e_z\big).
\end{align}
%
The classical X-ray transform in these coordinates for functions on a set $\Omega \subset B_1$ in $\RR^3$ is given by
%
\begin{align}
 &\OPP: L^2(\Omega) \Lto L^2(T \times U), \notag \\
 \label{eq:parflat:fwdproj}
 &\OPP f(\phi, u) = \int_\RR f\big(s\theta(\phi) + u_1 \theta'(\phi) + u_2 e_z\big)\, \D s,
\end{align}
%
and its adjoint is the backprojection
%
\begin{equation}
 \label{eq:parflat:backproj}
 \DUALOPP g(x) = \int_{-\phi_0}^{\phi_0} g\big(\phi, \INNER{x}{\theta'(\phi)}, \INNER{x}{e_z}\big)\, \D \phi.
\end{equation}
%
In our common framework, the directional mapping is given by $N(\phi, u) = -\theta(\phi)$, which is the negative of the canonical normal 
$N = \eta_1 \times \eta_2$ with $\eta_j = \partial_{u_j} X / \ABS{\partial_{u_j} X}$. It is $P(\phi, u) = \lbrace N(\phi, u)\rbrace$.
The rays are straight lines, and their formula is (leaving out the ``$\,\widetilde{\ }\,$'' to simplify notation)
%
\begin{equation*}
 \gamma_\phi(u) = \big( s \mapsto X(\phi, u) + (2-s) N(\phi, u),\ s > 0 \big).
\end{equation*}
%
They start at the far end of the object support $\Omega$ and continue to the detector, hence the integration domain of the $s$ 
integral can be extended to $\RR$. Thus, we get the transform
%
\begin{align*}
 \OPP: L^2(B_1) & \Lto L^2(T \times U) \\
 \OPP f(\phi, u) 
 &= \int_{\gamma_\phi(u)} f\, \D \gamma \\
 &= \int_0^\infty f\big(X(\phi, u) + (1-s) N(\phi, u)\big)\, \D s \\
 &= \int_\RR f\big(s \theta(\phi) + \rho(\phi) \cdot \TRANSP{(u_1, 0, u_2)}\big)\, \D s,
\end{align*}
%
which is \eqref{eq:parflat:fwdproj}. To compare the adjoint obtained by the generic formula \eqref{eq:general:backproj_onedir} with 
\eqref{eq:parflat:backproj}, we observe that we can write $x \in \Omega$ as
%
\begin{align*}
 x 
 &= \INNER{x}{\theta(\phi)}\, \theta(\phi) + \big(x - \INNER{x}{\theta(\phi)}\, \theta(\phi)\big) \\
 &= \INNER{x}{\theta(\phi)}\, \theta(\phi) + \INNER{x}{\theta'(\phi)}\, \theta'(\phi) + \INNER{x}{e_z}\, e_z \\
 &= \INNER{x}{\theta(\phi)}\, \theta(\phi) + \rho(\phi) \TRANSP{\big(\INNER{x}{\theta'(\phi)}, 0, \INNER{x}{e_z}\big)}.
\end{align*}
%
This means that for $s = 1 - \INNER{x}{\theta(\phi)}$ and $u = \TRANSP{\big(\INNER{x}{\theta'(\phi)}, \INNER{x}{e_z}\big)}$, it is 
$x = \gamma\big(\phi, u, N(\phi, u)\big)(s)$, and thus the mapping $\Gamma_\phi$ can be explicitly determined as
%
\begin{equation*}
 \Gamma_\phi(x) = \TRANSP{\big(1 - \INNER{x}{\theta(\phi)}, \INNER{x}{\theta'(\phi)}, \INNER{x}{e_z}\big)}
\end{equation*}
%
The Jacobian of this coordinate transform is apparently a column permutation of $\rho(\phi)$, hence the additional factors in the integral 
\eqref{eq:general:backproj_onedir} are one, and we can conclude that
%
\begin{equation*}
 \DUALOPP g(x) = \int_{-\phi_0}^{\phi_0} g\big(\phi, \INNER{x}{\theta'(\phi)}, \INNER{x}{e_z}\big)\, \D\phi,
\end{equation*}
%
which is the same as \eqref{eq:parflat:backproj}.%
\vspace{5ex}%



\subsection{Fan beam geometry}
\label{sec:applications:fanbeam}

In fan beam geometry, we consider objects described by a function supported in the unit ball $\Omega = B_1 \subset \RR^2$, a source moving 
on a circle with radius $r > 1$ and the detector moving on another circle with radius $R > 1$ such that source and detector always have 
maximum distance $R+r$ to each other, and the ray to the detector center is perpendicular to the detector line.

\subsubsection{Flat detector}
\label{sec:applications:fanbeam:flat}

A flat detector is described with the parameters
%
\begin{equation}
 \label{eq:fanflat:params}
 T = [0, \phi_0], \quad U = [-\tau_0, \tau_0], \quad 0 < \phi_0 \leq 2\pi,\ \tau_0 > 0,
\end{equation}
%
and has the parametrization
%
\begin{equation}
 \label{eq:fanflat:detector_parametr_alt}
 X(\phi, \tau) = R \theta(\phi) + \tau \theta'(\phi)
\end{equation}
%
with $\theta'(\phi) = \theta(\phi + \pi/2)$. The relation between the detector coordinate $\tau$ and the angle $\psi$ between the 
central ray and the ray from the source at $-r \theta(\phi)$ to the detector position $X(\phi,\tau)$ is given by the relation
%
\begin{equation*}
 \tan\alpha = \frac{\tau}{R + r},
\end{equation*}
%
hence the directional field $N$ is given by
%
\begin{equation}
 \label{eq:fanflat:dirfield}
 N(\phi, \tau) = -\theta\big(\phi + \alpha(\tau)\big), \quad \alpha(\tau) = \arctan\left( \frac{\tau}{R+r} \right) .
\end{equation} 
%
The forward model of the fan beam geometry for flat detector is the divergent beam transform
%
\begin{align}
 &\OPD: L^2(\Omega) \Lto L^2(T \times U) \notag \\
 \label{eq:fanflat:fwdproj}
 &\OPD f(\phi, \tau) = \int_0^\infty f\left(-r \theta(\phi) + s \theta\big(\phi + \alpha(\tau)\big) \right)\, \D s.
\end{align}
%
To relate this operator to the parallel beam X-ray transform, we calculate
%
\begin{equation*}
 \INNER{\theta(\phi)}{\theta'(\phi+\alpha)} = \cos(\alpha + \pi/2) = -\sin\alpha,
\end{equation*}
%
which implies
%
\begin{align*}
 \OPD f(\phi, \tau) 
 &= \int_\RR f \left( -r \theta(\phi) + s \theta\big(\phi + \alpha(\tau)\big) \right)\, \D s \\
 &= \int_\RR f \left( s \theta\big(\phi + \alpha(\tau)\big) + r\sin\alpha(\tau)\, \theta'\big(\phi + \alpha(\tau)\big) \right)\, 
 \D s \\
 &= \OPP f \big(\phi + \alpha(\tau), r\sin\alpha(\tau)\big) \\
 &= \OPU \OPP f(\phi, \tau)
\end{align*}
%
with the 2D parallel beam X-ray transform
%
\begin{align}
 &\OPP: L^2(B_1) \Lto L^2([0,2\pi) \times [-r,r]) \notag \\
 \label{eq:fanflat:xray_fwdproj}
 &\OPP f(\phi, \tau) = \int_\RR f\big(s\theta(\phi) + \tau \theta'(\phi)\big)\, \D s,
\end{align} 
%
and the coordinate transform 
%
\begin{align}
 &\OPU: L^2\big([0,2\pi) \times [-1,1]\big) \Lto L^2\big([0,\phi_0] \times [-\tau_0,\tau_0]\big) \notag \\
 \label{eq:fanflat:unit_trafo}
 &\OPU g(\phi, \tau) = g\big(\phi + \alpha(\tau), r\sin\alpha(\tau)\big).
\end{align} 
%
To calculate the backprojection $\OPD^*$, we determine the adjoint of $\OPU$:
%
\begin{align*}
 \INNER{\OPU g}{h}_{L^2([0,\phi_0] \times [-\tau_0,\tau_0])}
 &= \int_0^{\phi_0} \int_{-\tau_0}^{\tau_0} \OPU g(\phi, \tau)\, h(\phi, \tau)\, \D\tau\, \D\phi \\
 &= \int_0^{2\pi} \int_{-\tau_0}^{\tau_0} g\big(\phi + \alpha(\tau), r\sin\alpha(\tau)\big)\, h_0(\phi, \tau)\, \D\tau\, \D\phi \\
 &= \int_0^{2\pi} \int_{-\tau_0}^{\tau_0} g\big(\phi, r\sin\alpha(\tau)\big)\, h_0(\phi - \alpha(\tau), \tau)\, \D\tau\, \D\phi,
\end{align*}
%
where $h_0$ is the continuation of $h$ by zero to $[0,2\pi)\times\RR$. The coordinate change $s = r\sin\alpha(\tau)$, has the 
inverse
%
\begin{equation*}
 \tau = \frac{(R+r)s}{\sqrt{r^2-s^2}},
\end{equation*}
%
and further
%
\begin{equation*}
 \ABSLR{\frac{\D\tau}{\D s}} = \frac{r^2(R+r)}{(r^2-s^2)^{\nicefrac{3}{2}}}.
\end{equation*}
%
Thus, we get
%
\begin{align*}
 \INNER{\OPU g}{h}
 &= \int_0^{2\pi} \int_{-s_0}^{s_0} g\big(\phi, s\big)\, h_0\left( \phi - \arcsin(s/r), \frac{(R+r)s}{\sqrt{r^2-s^2}} \right)\, 
 \frac{r^2(R+r)}{(r^2-s^2)^{\nicefrac{3}{2}}}\, \D s\, \D\phi,
\end{align*}
%
with $s_0 = r\tau_0 / \sqrt{(R+r)^2 + \tau_0^2}$. If $s_0\geq 1$, i.e.
%
\begin{equation}
 \label{eq:fanflat:condition_det_width}
 \tau_0 \geq \frac{R + r}{\sqrt{r^2-1}},
\end{equation}
%
the inner integral extends over $[-1,1]$, such that the adjoint operator is
%
\begin{equation}
 \label{eq:fanflat:unit_trafo_adj}
 \OPU^* h(\phi, s) = \frac{r^2(R+r)}{(r^2-s^2)^{\nicefrac{3}{2}}}\, h_0\left( \phi - \arcsin(s/r), \frac{(R+r)s}{\sqrt{r^2-s^2}} \right).
\end{equation} 
%
Otherwise, a cutoff function $\chi_{[-s_0,s_0]}$ has to be added as an additional factor. This formula can be inserted as an argument to
$\OPP^*$, resulting in
%
\begin{equation*}
 \OPD^*g(x) = \OPP^*\OPU^*g(x) = \int_0^{2\pi} \OPU^*g\big(\phi, \INNER{x}{\theta'(\phi)}\big)\, \D\phi,
\end{equation*}
%
hence
%
\begin{equation}
 \label{eq:fanflat:backproj}
 \OPD^*g(x) = \int_0^{2\pi} w_{\text{ff}}(s)\, g_0\left( \phi - \arcsin(s/r), \frac{(R+r)s}{\sqrt{r^2-s^2}} \right)\, \D\phi
\end{equation} 
%
with
%
\begin{equation}
 \label{eq:fanflat:backproj_s}
 s = s(x; \phi) = \INNER{x}{\theta'(\phi)}
\end{equation}
%
and the integration weight
%
\begin{equation}
 \label{eq:fanflat:backproj_weight}
 w_{\text{ff}}(s) = \frac{r^2(R+r)}{(r^2-s^2)^{\nicefrac{3}{2}}}
\end{equation}
%
for the fan beam geometry with flat detector.
%
\vspace{5ex}

\subsubsection{Curved detector}
\label{sec:applications:fanbeam:curved}

Now we consider a detector with constant curvature, i.e. a part of a circle with radius $\rho \geq r$ centered at $-(\rho-R)\theta(\phi)$. 
It is parametrized over
%
\begin{equation*}
 T = [0, \phi_0],\quad U = [-\psi_0, \psi_0]
\end{equation*}
%
and can be written as
%
\begin{equation}
 \label{eq:fancurved:detector_parametr}
 X(\phi,\psi) = -(\rho-R) \theta(\phi) + \rho \theta(\phi + \psi).
\end{equation} 
%
To determine the directional field, we need the angle $\beta$ between the central ray and the line from the source $x_{\text{s}} = 
-r\theta(\phi)$ to a detector point $X(\phi,\psi)$. It is
%
\begin{equation*}
 \cos\beta = \frac{\INNER{X(\phi,\psi)-x_{\text{s}}}{\theta(\phi)}}{\ABSLR{X(\phi,\psi)-x_{\text{s}}}}
 = \frac{R+r-\rho + \rho\cos\psi}{\sqrt{(R+r-\rho)^2 + \rho^2 + 2\rho(R+r-\rho)\cos\psi}},
\end{equation*}
%
i.e.
%
\begin{equation}
 \label{eq:fancurved:direction_field_angle}
 \beta(\psi) = \arccos \left( \frac{R+r-\rho(1-\cos\psi)}{\sqrt{(R+r-\rho)^2 + \rho^2 + 2\rho(R+r-\rho)\cos\psi}} \right)
\end{equation} 
%
and
\begin{equation}
 \label{eq:fancurved:direction_field}
 N(\phi,\psi) = -\theta\big(\phi + \beta(\psi)\big).
\end{equation} 
%
The relation between the parameter $\psi$ of the curved detector and the variable $\tau$ in the flat geometry is $\tau = \rho\tan\psi$, 
thus the divergent beam transform
%
\begin{align}
 &\OPD: L^2(\Omega) \Lto L^2(T \times U), \notag \\
 \label{eq:fancurved:fwdproj}
 &\OPD f(\phi, \psi) = \int_0^\infty f\left(-r \theta(\phi) + s \theta\big(\phi + \beta(\psi)\big) \right)\, \D s,
\end{align}
%
relates to the projection $\OPD_{\text{f}}$ of the flat geometry as
%
\begin{equation*}
 \OPD f(\phi,\psi) = \OPD_{\text{f}} f(\phi, \rho\tan\psi) = \OPT \OPD_{\text{f}}(\phi,\psi)
\end{equation*} 
%
with the coordinate transform
%
\begin{align}
 & \OPT: L^2([0,\phi_0]\times \RR) \Lto L^2([0,\phi_0] \times [-\psi_0, \psi_0]), \notag \\
 \label{eq:fancurved:coord_op}
 & \OPT g(\phi, \psi) = g(\phi, \rho\tan\psi).
\end{align}
%
Its adjoint is determined via the change of variables
%
\begin{equation*}
 \tau = \rho\tan\psi,\quad \frac{\D\tau}{\D\psi} = \rho (1 + \tan^2\psi) = \frac{\rho^2 + \tau^2}{\rho}
\end{equation*}
%
as
%
\begin{equation}
 \label{eq:fancurved:coord_op_adjoint}
 \OPT^* g(\phi, \tau) = g_0\big(\phi, \arctan(\tau/\rho)\big)\, \frac{\rho}{\rho^2 + \tau^2},
\end{equation} 
%
where $g_0$ is the continuation of $g$ to $[0,\phi_0] \times \RR$ by zero. Hence, the backprojection in curved coordinates is
%
\begin{equation*}
 \OPD^*g(x) = \int_0^{2\pi} w_{\text{ff}}(s)\, (\OPT^*g)_0\left( \phi - \arcsin(s/r), 
 \frac{(R+r)s}{\sqrt{r^2-s^2}} \right)\, \D\phi
\end{equation*}
%
with $s = s(x; \phi)$ according to \eqref{eq:fanflat:backproj_s}. Inserting $\tau = (R+r)s/\sqrt{r^2-s^2}$ into the factor 
$\rho/(\rho^2+\tau^2)$ and multiplying with $w_{\text{ff}}$ yields the total weight 
%
\begin{equation}
 \label{eq:fancurved:backproj_weight}
 w_{\text{fc}}(s) = \frac{\rho r^2(R+r)}{\sqrt{r^2-s^2}\left( \rho^2 r^2 + \big((R+r)^2 - \rho^2\big) s^2 \right)}
\end{equation}
%
for the curved geometry. The backprojection thus is
%
\begin{equation}
 \label{eq:fancurved:backproj}
 \OPD^*g(x) = \int_0^{2\pi} w_{\text{fc}}(s)\, g_0\left( \phi - \arcsin(s/r), 
 \arctan \left(\frac{(R+r)s}{\rho\sqrt{r^2-s^2}} \right) \right)\, \D\phi
\end{equation} 
%
with $s = s(x; \phi) = \INNER{x}{\theta'(\phi)}$. The condition on $\tau_0$ reads as
%
\begin{equation}
 \label{eq:fancurved:condition_det_width}
 \psi_0 \geq \arctan\left( \frac{R+r}{\rho\sqrt{r^2-1}} \right)
\end{equation}
%
here.
\vspace{5ex}%



%\vspace{5ex}%



\subsection{Cone beam geometry -- circular acquisition}
\label{sec:applications:cone_circular}

Cone beam geometry is the 3D equivalent to the 2D fan beam geometry. Many more different acquisition curves can be realized due to the 
larger flexibility in three dimensions. We consider first the circular acquisition case, where source and detector move on circles with 
radii $r$ or $R$, respectively, in a common 2D plane. The detector is now also two-dimensional, so different combinations of curvature can 
be considered. Objects are assumed to be supported in $\Omega = B_1 \subset \RR^3$.

\subsubsection{Flat detector}
\label{sec:applications:cone_circular:flat}

We start with the flat detector as used mostly in nondestructive testing. We parametrize it over the sets
%
\begin{equation}
 \label{eq:coneflat:params}
 T = [0, \phi_0], \quad U = [-\tau_0, \tau_0] \times [-z_0, z_0]
\end{equation}
%
by
%
\begin{equation}
 \label{eq:coneflat:detector_parametr}
 X(\phi, \tau, z) = r\theta(\phi) + \tau\, \theta'(\phi) + z\, e_z,
\end{equation} 
%
using the unit vectors
%
\begin{equation}
 \label{eq:coneflat:unit_vectors}
 \theta(\phi) = \TRANSP{(\cos\phi, \sin\phi, 0)}, \quad \theta'(\phi) = \TRANSP{(-\sin\phi, \cos\phi, 0)}, \quad e_z = \TRANSP{(0,0,1)}.
\end{equation}
%
Hence, the rotation axis is the $z$ axis. Since the azimuthal and vertical components can be treated independently, we can conclude from 
the 2D case that the directional field is given by
%
\begin{equation}
 \label{eq:coneflat:direction_field}
 N(\phi, \tau, z) = - \omega\big(\phi + \alpha(\tau), \alpha(z)\big), \quad \alpha(\tau) = \arctan\left( \frac{\tau}{R+r} \right)
\end{equation}
%
(cf. \eqref{eq:fanflat:dirfield}) and the unit vector
%
\begin{equation}
 \label{eq:coneflat:unit_vec_3d}
 \omega(\phi, \vartheta) = \TRANSP{(\cos\vartheta\cos\phi, \cos\vartheta\sin\phi, \sin\vartheta)} = \cos\vartheta\, \theta(\phi) + 
 \sin\vartheta\, e_z.
\end{equation} 
%
Following these considerations, we define the forward projection
%
\begin{align}
 &\OPD: L^2(\Omega) \Lto L^2(T \times U) \\
 \label{eq:coneflat:fwdproj}
 &\OPD f(\phi, \tau, z) = \int_0^\infty f\left( -r\theta(\phi) + s \omega\big(\phi + \alpha(\tau), \alpha(z)\big) \right) \D s.
\end{align}
%
As in the 2D case, we want to use the relation to the parallel beam X-ray transform. In three dimensions, this operator is defined as
%
\begin{align}
 & \OPP: L^2(\Omega) \Lto L^2([0,2\pi) \times [-\pi/2, \pi/2] \times \RR^2) \notag \\
 \label{eq:coneflat:xray_fwdproj}
 & \OPP f(\phi, \vartheta, s, v) = \int_\RR f\left( t\omega(\phi, \vartheta) + s \theta'(\phi) + v \omega'(\phi,\vartheta) 
 \right)\, \D t, 
\end{align}
%
where $\omega$, $\theta'$ and $\omega'(\phi, \vartheta) = -\sin\vartheta\, \theta(\phi) + \cos\vartheta\, e_z$ form the natural 
right-handed coordinate system on the unit sphere. The adjoint of $\OPP$ is given by
%
\begin{equation}
 \label{eq:coneflat:xray_backproj}
 \OPP^* g(x) = \int_0^{2\pi} \int_{-\pi/2}^{\pi/2} g\left( \phi, \vartheta, \INNER{x}{\theta'(\phi)}, \INNER{x}{\omega'(\phi, \vartheta} 
 \right) \D\vartheta\, \D\phi.
\end{equation}
%
To relate $\OPD$ to $\OPP$, we calculate
%
\begin{align*}
 \INNER{\theta(\phi)}{\omega'(\phi + \alpha_1, \alpha_2)} 
 &= -\sin\alpha_2 \INNER{\theta(\phi)}{\theta(\phi + \alpha_1)} + \cos\alpha_2 \INNER{\theta(\phi)}{e_z} \\
 &= -\cos\alpha_1 \sin\alpha_2.
\end{align*}
%
Using the result $\INNER{\theta(\phi)}{\theta'(\phi+\alpha_1)} = -\sin\alpha_1$ from the 2D case, we acquire the relation
%
\begin{align*}
 \OPD f(\phi, \tau, z) 
 &= \OPP f\left( \phi + \alpha(\tau), \alpha(z), r\sin\alpha(\tau), r\cos\alpha(\tau) \sin\alpha(z) \right) \\
 &= \OPU \OPP f(\phi, \tau, z)
\end{align*}
%
with the coordinate transform operator
%
\begin{align}
 & \OPU: L^2([0,2\pi) \times [-\pi/2, \pi/2] \times B_1) \Lto L^2([0,\phi_0] \times [-\tau_0, \tau_0] \times [-z_0, z_0]) \notag \\
 & \OPU g(\phi, \tau, z) = g\left( \phi + \alpha(\tau), \alpha(z), r\sin\alpha(\tau), r\cos\alpha(\tau) \sin\alpha(z) \right),
\end{align}
%
where $B_1 \subset \RR^2$. Note that this operator is unbounded in the $L^2$ sense since it reduces 
dimensionality by 1. However, its composition with $\OPP$ is bounded in $L^2$, such that we can proceed formally and ensure that the final 
composed expression for the adjoint is well-defined. We calculate
%
\begin{align*}
 \INNER{\OPU g}{h} 
 &= \int_0^{\phi_0} \int_{-\tau_0}^{\tau_0} \int_{-z_0}^{z_0} \OPU g(\phi, \tau, z)\, h(\phi, \tau, z)\, \D z\, \D\tau\, \D\phi \\
 &= \int_0^{2\pi} \int_{-\tau_0}^{\tau_0} \int_{-z_0}^{z_0} g\left( \phi + \alpha(\tau), \alpha(z), r\sin\alpha(\tau), r\cos\alpha(\tau) 
 \sin\alpha(z) \right)\, h_0(\phi, \tau, z)\, \D z\, \D\tau\, \D\phi \\
 &= \int_0^{2\pi} \int_{-\tau_0}^{\tau_0} \int_{-z_0}^{z_0} g\left( \phi, \alpha(z), r\sin\alpha(\tau), r\cos\alpha(\tau) 
 \sin\alpha(z) \right)\, h_0\left( \phi - \alpha(\tau), \tau, z \right)\, \D z\, \D\tau\, \D\phi \\
 &= \int_0^{2\pi} \int_{-\tau_0}^{\tau_0} \int_{-\vartheta_0}^{\vartheta_0} g\left( \phi, \vartheta, r\sin\alpha(\tau), r\cos\alpha(\tau) 
 \sin\vartheta \right) \cdot \phantom{x} \\ 
 & \hspace{80pt} h_0\left( \phi - \alpha(\tau), \tau, (R+r)\tan\vartheta \right)\, (R+r)(1+\tan^2\vartheta)\, \D \vartheta\, \D\tau\, 
 \D\phi,
\end{align*}
%
with $\vartheta_0 = \arctan(z_0/(R+r))$, where the change of variables $z = (R+r)\tan\vartheta$ was applied. With $h_0$, we denote 
analogously to the 2D case the continuation of $h$ by zero to $[0,2\pi)\times\RR^2$. Next, the same variable change $s=r\sin\alpha(\tau)$ as 
for \eqref{eq:fanflat:unit_trafo} leads to
%
\begin{align*}
 \INNER{\OPU g}{h} 
 = \int_0^{2\pi} \int_{-s_0}^{s_0} \int_{-\vartheta_0}^{\vartheta_0} & g\left( \phi, \vartheta, s, \sqrt{r^2-s^2}\sin\vartheta \right)
 \cdot \phantom{x} \\ 
 & h_0\left( \phi - \arcsin(s/r), \frac{(R+r)s}{\sqrt{r^2-s^2}}, (R+r)\tan\vartheta \right) \cdot \phantom{x} \\
 &\frac{r^2(R+r)^2(1+\tan^2\vartheta)}{(r^2-s^2)^{\nicefrac{3}{2}}} \, \D \vartheta\, \D\tau\, \D\phi.
\end{align*}
%
Writing the point evaluation in the fourth argument of $g$ as an integral with the Delta distribution 
$\delta\big(v - \sqrt{r^2-s^2}\sin\vartheta\big)$, the adjoint of $\OPU$ can be identified as
%
\begin{align}
 \label{eq:coneflat:coord_op_adjoint}
 \OPU^*h(\phi,\vartheta,s,v) =\ &\delta\left( v - \sqrt{r^2-s^2}\sin\vartheta \right)\, 
 h_0\left( \phi - \arcsin(s/r), \frac{(R+r)s}{\sqrt{r^2-s^2}}, (R+r)\tan\vartheta \right) \cdot \phantom{x} \\
 &\frac{r^2(R+r)^2}{\cos^2\vartheta\, (r^2-s^2)^{\nicefrac{3}{2}}} \, \D \vartheta\, \D\tau\, \D\phi. \notag
\end{align}
%
If this expression is inserted into the formula for $\OPP^*$, the $\vartheta$ integration can be eliminated with the Delta distribution. To 
do this, the distribution must be expressed in terms of $\vartheta$ with the help of
%
\begin{equation*}
 \delta\big(w(\vartheta)\big) = \sum_{\vartheta:w(\vartheta)=0} \frac{\delta(\vartheta)}{\ABS{w'(\vartheta)}}
\end{equation*}
%
using
%
\begin{align*}
 w(\vartheta) 
 &= v(x; \phi, \vartheta) - \sqrt{r^2-s^2}\sin\vartheta \\
 &= -\sin\vartheta \INNER{x}{\theta'(\phi)} + \cos\vartheta \INNER{x}{e_z} - \sqrt{r^2-s^2}\sin\vartheta \\
 &= -\sin\vartheta\, \big(s + \sqrt{r^2-s^2}\big) + \cos\vartheta\, x_3.
\end{align*}

The only zero of this function is
%
\begin{equation*}
 \vartheta^* = \arctan\left( \frac{x_3}{s + \sqrt{r^2 - s^2}} \right).
\end{equation*}
%
With $\sin(\arctan x) = x/\sqrt{1+x^2}$ and $\cos(\arctan x) = 1/\sqrt{1+x^2}$, we have
%
\begin{align*}
 \sin\vartheta^* &= \frac{x_3}{\sqrt{x_3^2 + \big(s + \sqrt{r^2-s^2}\big)^2}} = \frac{x_3}{\sqrt{x_3^2 + r^2 + 2s\sqrt{r^2-s^2}}} \\
 \cos\vartheta^* &= \frac{\sqrt{r^2 + 2s\sqrt{r^2-s^2}}}{\sqrt{x_3^2 + r^2 + 2s\sqrt{r^2-s^2}}},
\end{align*}
%
such that the derivative of $w$ at $\vartheta^*$ can be evaluated as
%
\begin{align*}
 w'(\vartheta^*) 
 &= -s \cos\vartheta^* - x_3 \sin\vartheta^* \\
 &= - \frac{s \sqrt{r^2 + 2s\sqrt{r^2-s^2}} + x_3^2}{\sqrt{x_3^2 + r^2 + 2s\sqrt{r^2-s^2}}}
\end{align*}
%
Since $s^2+v^2 \leq 1$, the zero $\vartheta^*$ is contained in the $\vartheta$ integral if $\tan\vartheta_0 \geq 1/\sqrt{r^2-1}$ or, 
expressed with $z_0$,
%
\begin{equation}
 \label{eq:coneflat:condition_det_height}
 z_0 \geq \frac{R+r}{\sqrt{r^2-1}}.
\end{equation}
%
The same condition for $s_0$ can be applied from the 2D case, see \eqref{eq:fanflat:condition_det_width}. If one of the conditions is not 
fullfilled, the adjoint $\OPU^*$ has to be supplemented with a cutoff function in the corresponding variable. Finally, the adjoint $\OPD^*$ 
can be expressed as
%
\begin{align*}
 \OPD^* g(x)
 &= \int_0^{2\pi} \int_{-\pi/2}^{\pi/2} \OPU^* g\left( \phi, \vartheta, s(\phi,x), v(\phi,\vartheta, x) \right) \D\vartheta\, \D\phi \\
 &= \int_0^{2\pi} g_0 \left( \phi - \arcsin(s/r), \frac{(R+r)s}{\sqrt{r^2-s^2}}, 
 \frac{(R+r)x_3}{s + \sqrt{r^2-s^2}} \right) \cdot \phantom{x} \\ 
 &\hspace{35pt} \frac{r^2(R+r)^2}{(r^2-s^2)^{\nicefrac{3}{2}}}\, 
 \frac{\big(x_3^2 + r^2 + 2s\sqrt{r^2-s^2}\big)^{\nicefrac{3}{2}}}{\left( s \sqrt{r^2 + 2s\sqrt{r^2-s^2}} + x_3^2 \right)
 \left( r^2 + 2s\sqrt{r^2-s^2} \right)}\, \D\phi,
\end{align*}
%
hence
%
\begin{equation}
 \label{eq:coneflat:backproj}
 \OPD^* g(x) = \int_0^{2\pi} w_{\text{ccf}}(s)\, g_0 \left( \phi - \arcsin(s/r), \frac{(R+r)s}{\sqrt{r^2-s^2}}, 
 \frac{(R+r)x_3}{s + \sqrt{r^2-s^2}} \right)\, \D\phi
\end{equation} 
%
with the weight function
%
\begin{equation}
 \label{eq:coneflat:backproj_weight}
 w_{\text{ccf}}(s) =\frac{r^2(R+r)^2}{(r^2-s^2)^{\nicefrac{3}{2}}}\, 
 \frac{\big(x_3^2 + r^2 + 2s\sqrt{r^2-s^2}\big)^{\nicefrac{3}{2}}}{\left( s \sqrt{r^2 + 2s\sqrt{r^2-s^2}} + x_3^2 \right)
 \left( r^2 + 2s\sqrt{r^2-s^2} \right)}
\end{equation} 
%
of the circular cone beam geometry with flat detector and
%
\begin{align}
 s &= s(\phi; x) = \INNER{x}{\theta'(\phi)}, \\
 x_3 &= \INNER{x}{e_z}.
\end{align}
%
\vspace{5ex}



\subsubsection{Detector curved along the azimuth angle}
\label{sec:applications:cone_circular:curvedstack}

These detectors are probably the most common ones in medical imaging. They consist in curved line detectors stacked along the vertical 
axis, hence the results from the curved detectors in the fan beam geometry (section \ref{sec:applications:fanbeam:curved} can be applied. 
We assume again a curvature radius of $\rho > 0$ and use the parameter sets
%
\begin{equation}
 \label{eq:conestack:params}
 T = [0, \phi_0],\quad U = [-\psi_0, \psi_0] \times [-z_0, z_0], \quad 0 < \phi_0 < 2\pi,\ 0 < \psi_0 < \pi/2,\ z_0 > 0,
\end{equation}
%
to represent the detector as
%
\begin{equation}
 \label{eq:conestack:detector_parametr}
 X(\phi, \psi, z) = -(\rho - R)\, \theta(\phi) + \rho\, \theta\big((\phi + \alpha(\psi)\big) + z\, e_z.
\end{equation}
%
The directional field can be written, analogously to \eqref{eq:fancurved:direction_field}, as
%
\begin{equation}
 \label{eq:conestack:dirfield}
 N(\phi, \psi, z) = -\omega\big( \phi + \beta(\psi), \alpha(z) \big),
\end{equation} 
%
with $\alpha$ and $\beta$ defined as before. The corresponding forward projection is
%
\begin{align}
 &\OPD: L^2(\Omega) \Lto L^2(T \times U) \notag \\
 &\OPD f(\phi, \psi, z) = \int_0^\infty f\left( -r\theta(\phi) + s \omega\big( \phi + \beta(\psi), \alpha(z) \big) \right)\, \D s.
\end{align}
%
We use its relation to the operator $\OPD_{\text{f}}$ for flat detector as defined in \eqref{eq:coneflat:fwdproj} with the coordinate 
transform $\tau = \rho\tan\psi$, i.e. 
%
\begin{equation*}
 \OPD f(\phi, \psi, z) = \OPD_{\text{f}} f\big(\phi, \rho\tan\psi, z\big) = \OPT \OPD_{\text{f}} f(\phi, \psi, z)
\end{equation*}
%
where $\OPT$ acts as the transform \eqref{eq:fancurved:coord_op} in the second variable. Thus, by inserting the adjoint analogous 
to \eqref{eq:fancurved:coord_op_adjoint} into the formula \eqref{eq:coneflat:backproj} for $\OPD_{\text{f}}^*$, we get the backprojection
%
\begin{equation}
 \label{eq:conestack:backproj}
 \OPD^* g(x) = \int_0^{2\pi} w_{\text{cca}}(s)\, g_0 \left( \phi - \arcsin(s/r), \arctan\left( \frac{(R+r)s}{\rho\sqrt{r^2-s^2}} \right), 
 \frac{(R+r)v}{\sqrt{r^2-s^2-v^2}} \right)\, \D\phi
\end{equation} 
%
with the weight function
%
\begin{equation}
 \label{eq:conestack:backproj_weight}
 w_{\text{cca}}(s) = \frac{\rho r^2(R+r)^2\sqrt{r^2-s^2}}{\left( \rho^2 r^2 + \big((R+r)^2 - \rho^2\big) s^2 \right)}
\end{equation}
%
for the \textbf{c}ircular \textbf{c}one beam geometry with \textbf{a}zimuthally curved detector and $s=s(x;\phi), v=v(x;\phi,\vartheta)$ as 
defined in .

\vspace{5ex}%



\subsubsection{Two-fold curved detector}
\label{sec:applications:cone_circular:twofoldcurved}

In this setting, we consider a detector which is curved both horizontally and vertically, but with possibly different curvatures. Let
%
\begin{equation}
 \label{eq:conecurved:unitvecs}
 \theta(\phi) = \TRANSP{(\cos\phi, \sin\phi, 0)}, \quad \theta(\phi) = \TRANSP{(-\sin\phi, \cos\phi, 0)}, \quad e_z = \TRANSP{(0, 0, 1)}.
\end{equation}
%
be the unit vectors in the rotating system. The detector is parametrized over the sets
%
\begin{equation}
 \label{eq:conecurved:params}
 T = [0, \phi_0],\quad U = [-\psi_0, \psi_0] \times [-\vartheta_0, \vartheta_0],
\end{equation}
%
and is assumed to have curvature $1/R_1$ along $\theta'(\phi)$ and $1/R_2$ along $e_z$ with $R_1,R_2 >0$. From the 2D case, we know that 
for $\theta=0$, the detector point associated with $\phi$ and $\psi$ is
%
\begin{equation*}
 d_1 = c_1 + R_1 \theta(\phi + \psi), \quad c_1 = -(R_1 - r) \theta(\phi).
\end{equation*}
%
If $\theta \neq 0$, the detector point is given by a second rotation applied to $d_1$, the center of rotation being $c_2 = -(R_2 - r) 
\theta(\phi)$ and the rotation axis being $-\theta'(\phi)$. This rotation can be generally expressed as
%
\begin{equation*}
 x \mapsto c_2 + \rho(x - c_2), \quad \rho(x) = \INNER{x}{\theta'}\, \theta' + \cos\vartheta \big(x - \INNER{x}{\theta'}\, \theta'\big) 
 - \sin\vartheta\, (\theta' \times x)
\end{equation*}
%
with the cross product $\times$ in $\RR^3$. Using the relations
%
\begin{align*}
 \theta'(\phi) \times \theta(\phi) &= -e_z, \\
 e_z \times \theta'(\phi) &= -\theta(\phi), \\
 \INNER{\theta(\phi + \psi)}{\theta'(\phi)} &= \sin\psi \\
 \theta(\phi + \psi) - \INNER{\theta(\phi + \psi)}{\theta'(\phi)}\, \theta'(\phi) &= \cos\psi\, \theta(\phi)\\
 \theta'(\phi) \times \theta(\phi + \psi) &= -\cos\psi\, e_z.
\end{align*}
%
and inserting $x = d_1$, we get the detector point
%
\begin{align}
 X(\phi, \psi, \theta)
 &= c_2 + \rho(d_1 - c_2) \notag \\
 &= -(R_2 - r) \theta(\phi) + \rho\big(c_1 + R_1 \theta(\phi + \psi) - c_2\big) \notag \\
 &= -(R_2 - r) \theta(\phi) + \rho\big((R_2-R_1) \theta(\phi) + R_1 \theta(\phi + \psi)\big) \notag \\
 &= -(R_2 - r) \theta(\phi) + (R_2-R_1) \rho\big(\theta(\phi)\big) + R_1 \rho\big(\theta(\phi + \psi)\big) \notag \\
 &= -(R_2 - r) \theta(\phi) + (R_2-R_1) \big[ \cos\vartheta\, \theta(\phi) + \sin\vartheta\, e_z \big] + \phantom{x} \notag \\
 &\hspace{25pt} R_1 \big[ \sin\psi\, \theta'(\phi) + \cos\vartheta\, \cos\psi\, \theta(\phi) + \sin\vartheta\, \cos\psi\, e_z \big] \notag 
 \\
 &= \big[ -(R_2 - r) + (R_2 - R_1) \cos\vartheta + R_1 \, \cos\vartheta\, \cos\psi \big]\, \theta(\phi) + \phantom{x} \notag \\
 &\hspace{14pt} \big[R_1 \sin\psi\big]\, \theta'(\phi) + \big[ (R_2 - R_1) \sin\vartheta + R_1\, \sin\vartheta\, \cos\psi \big]\, e_z 
\notag  \\
 \label{eq:conecurved:detector_parametr}
 &= \big[ r - 2R_2 \sin^2(\vartheta/2) - 2R_1 \cos\vartheta \sin^2(\psi/2) \big]\, \theta(\phi) + \big[R_1 \sin\psi\big]\, 
 \theta'(\phi) + \phantom{x} \\
 &\hspace{14pt} \big[ -2R_1 \sin\vartheta\sin^2(\psi/2) + R_2\sin\vartheta \big]\, e_z \notag 
\end{align}
%
\NOTE{Check if this makes sense -- especially if the $z$ component can really depend on $\psi$!}%
The two rotations with respect to the angles $\psi$ and $\vartheta$ are independent, thus the result from the 2D case on the relation 
between detector angles and angles as seen from the source point holds for both $\psi$ and $\vartheta$ separately. It follows immediately 
that the directions from the detector to the source are given by
%
\begin{equation}
 \label{eq:conecurved:direction_field}
 N(\phi, \psi, \vartheta) = -\omega(\phi + \beta_1\psi/2, \beta_2\vartheta/2)
\end{equation} 
%
with $R_j = \beta_j r,\ j=1,2$ and the unit vector
%
\begin{equation}
 \label{eq:conecurved:unitvec_omega}
 \omega(\phi, \theta) = \TRANSP{(\cos\vartheta \cos\phi, \cos\vartheta\sin\phi, \sin\vartheta)}.
\end{equation} 
%
Now we define the 3D divergent beam transform for functions on $\Omega = B_1 \subset \RR^3$:
%
\begin{align}
 &\OPD: L^2(\Omega) \Lto L^2(T \times U) \notag \\
 \label{eq:conecurved:fwdproj}
 &\OPD f(\phi, \psi, \vartheta) = \int_0^\infty f\big( -r\theta(\phi) + s\omega(\phi + \beta_1\psi/2, \beta_2\vartheta/2) \big)\, \D s.
\end{align}
%
To relate this operator to the parallel beam X-ray transform, we need to calculate the projection of $\theta(\phi)$ onto
$\omega(\phi + \beta_1\psi/2, \beta_2\vartheta/2)$. For angles $\phi$ and $\vartheta$, the canonical basis in 
$\omega(\phi, \vartheta)^\perp$ is given by the normalized derivatives with respect to $\phi$ and $\vartheta$, respectively. These vectors 
are
%
\begin{align}
 \label{eq:conecurved:omega1perp}
 \omega_1^\perp(\phi, \vartheta) &= \theta'(\phi), \\
 \label{eq:conecurved:omega2perp}
 \omega_2^\perp(\phi, \vartheta) &= -\sin\vartheta\, \theta(\phi) + \cos\vartheta\, e_z.
\end{align}
%
Thus, we can write
%
\begin{align*}
 \Pi_{\omega(\phi + \beta_1\psi/2, \beta_2\vartheta/2)} \theta(\phi)
 &= \INNER{\theta(\phi)}{\theta'(\phi + \beta_1\psi/2)}\, \omega_1^\perp + \phantom{x} \\
 &\hspace{14pt} \INNER{\theta(\phi)}{-\sin(\beta_2\vartheta/2)\, \theta(\phi + \beta_1\psi/2) + \cos(\beta_2\vartheta/2)\, e_z}\, 
 \omega_2^\perp \\
 &= -\sin(\beta_1\psi/2)\, \omega_1^\perp - \cos(\beta_1\psi/2) \sin(\beta_2\vartheta/2)\, \omega_2^\perp.
\end{align*}
%
In consequence,
%
\begin{align*}
 \OPD f(\phi, \psi, \vartheta)
 &= \int_\RR f\big( s\omega(\phi + \beta_1\psi/2, \beta_2\vartheta/2) + r\sin(\beta_1\psi/2)\, \omega_1^\perp + 
 r \cos(\beta_1\psi/2) \sin(\beta_2\vartheta/2)\, \omega_2^\perp \big)\, \D s \\
 &= \OPP f\big(\phi + \beta_1\psi/2, \beta_2\vartheta/2, r\sin(\beta_1\psi/2), r \cos(\beta_1\psi/2) \sin(\beta_2\vartheta/2) \big) \\
 &= \OPU \OPP f(\phi, \psi, \vartheta)
\end{align*}
%
with the 3D X-ray transform
%
\begin{align}
 &\OPP: L^2(\Omega) \Lto L^2\big( [0, 2\pi) \times [-\pi/2, \pi/2] \times \RR^2 \big) \notag \\
 \label{eq:conecurved:xray_proj}
 &\OPP f(\phi, \vartheta, \sigma, \tau) = \int_\RR f(s \omega(\phi, \vartheta) + \sigma \omega_1^\perp + \tau \omega_2^\perp \big)\, \D s
\end{align}
%
and the coordinate transform
%
\begin{align}
 &\OPU: L^2\big( [0, 2\pi) \times [-\pi/2, \pi/2] \times \RR^2 \big) \Lto L^2([0, \phi_0] \times [-\psi_0, \psi_0] \times 
 [-\vartheta_0, \vartheta_0] \big) \notag \\
 \label{eq:conecurved:coord_op}
 &\OPU g(\phi, \psi, \vartheta) = g\big( \phi + \beta_1\psi/2, \beta_2\vartheta/2, r\sin(\beta_1\psi/2), r \cos(\beta_1\psi/2) 
 \sin(\beta_2\vartheta/2) \big).
\end{align}
%
The adjoint of $\OPP$ is the usual parallel beam backprojection
%
\begin{equation}
 \label{eq:conecurved:xray_backproj}
 \DUALOPP g(x) = \int_0^{2\pi} \int_{-\pi/2}^{\pi/2} g\big( \phi, \vartheta, \INNER{x}{\omega_1^\perp(\phi, \vartheta)}, 
 \INNER{x}{\omega_2^\perp(\phi, \vartheta)} \big)\, \D\vartheta\, \D\phi.
\end{equation} 
%
For the operator $\OPU$, we calculate
%
\begin{align*}
 \INNER{\OPU g}{h}
 &= \int_0^{\phi_0} \int_{-\psi_0}^{\psi_0} \int_{-\vartheta_0}^{\vartheta_0} g\big( \phi + \beta_1\psi/2, \beta_2\vartheta/2, 
 r\sin(\beta_1\psi/2), r \cos(\beta_1\psi/2) \sin(\beta_2\vartheta/2) \big) \cdot \phantom{x} \\
 &\hspace{80pt} h(\phi, \psi, \vartheta)\, \D\vartheta\, \D\psi\, \D\phi \\
 &= \int_0^{2\pi} \int_{-\psi_0}^{\psi_0} \int_{-\beta_2\vartheta_0/2}^{\beta_2\vartheta_0/2} g\big( \phi, \vartheta, 
 r\sin(\beta_1\psi/2), r \cos(\beta_1\psi/2) \sin\vartheta \big) \cdot \phantom{x} \\
 &\hspace{95pt} h_0\big( \phi - \beta_1\psi/2, \psi, 2\vartheta/\beta_2 \big)\, 
 \frac{2}{\beta_2}\, \D\vartheta\, \D\psi\, \D\phi \\
 &= \int_0^{2\pi} \int_{-r\sin(\beta_1\psi_0/2)}^{r\sin(\beta_1\psi_0/2)} \int_{-\beta_2\vartheta_0/2}^{\beta_2\vartheta_0/2}
 g\big( \phi, \vartheta, \sigma, \sqrt{r^2 - \sigma^2} \sin\vartheta \big) \cdot \phantom{x} \\
 &\hspace{135pt} h_0\big( \phi - \arcsin(\sigma/r), 2\arcsin(\sigma/r)/\beta_1, 2\vartheta/\beta_2 \big)\, \cdot \phantom{x} \\
 &\hspace{135pt} \frac{4}{\beta_1\beta_2\sqrt{r^2 - \sigma^2}}\, \D\vartheta\, \D\psi\, \D\phi \\
 &= \int_0^{2\pi} \int_{-r\sin(\beta_1\psi_0/2)}^{r\sin(\beta_1\psi_0/2)} \int_{-\beta_2\vartheta_0/2}^{\beta_2\vartheta_0/2} \int_{-r}^r
 g(\phi, \vartheta, \sigma, \tau)\, \delta\big(\tau - \sqrt{r^2 - \sigma^2} \sin\vartheta \big) \cdot \phantom{x} \\
 &\hspace{135pt} h_0\big( \phi - \arcsin(\sigma/r), 2\arcsin(\sigma/r)/\beta_1, 2\vartheta/\beta_2 \big) \cdot \phantom{x} \\
 &\hspace{135pt} \frac{4}{\beta_1\beta_2\sqrt{r^2 - \sigma^2}}\, \D\tau\, \D\vartheta\, \D\sigma\, \D\phi.
\end{align*}
%
The adjoint of $\OPU$ in the sense of unbounded operators can be read off the last identity as
%
\begin{align}
 \label{eq:conecurved:coord_op_adjoint}
 \DUALOPU g(x) 
 &= \frac{4}{\beta_1\beta_2\sqrt{r^2 - \sigma^2}}\,
 h_0\big( \phi - \arcsin(\sigma/r), 2\arcsin(\sigma/r)/\beta_1, 2\vartheta/\beta_2 \big) \cdot \phantom{x} \\
 &\hspace{15pt}\delta\big(\tau - \sqrt{r^2 - \sigma^2} \sin\vartheta \big) \notag.
\end{align} 
%
Since the composition $\OPU \OPP$ is known to 
be bounded, its adjoint is given by $\DUALOPP \DUALOPU$. The $\vartheta$ integration in \eqref{eq:conecurved:xray_backproj} can be reduced 
with the $\delta$ distribution if for each $\tau$ and each $\sigma$ in the respective intervals, there is a $\vartheta$ in the integration 
domain such that $\tau - \sqrt{r^2 - \sigma^2} \sin\vartheta = 0$. Since we insert $\sigma = \INNER{x}{\omega_1^\perp}$ and $\tau = 
\INNER{x}{\omega_2^\perp}$, $\ABS{\tau_{\text{max}}} = 1$ and the minimum of the root is $\sqrt{r^2 - 1}$ or 
$\sqrt{r^2 - \sigma_{\text{max}}^2} = r\cos(\beta_1\psi_0/2)$. Thus, $\vartheta_0$ has to fulfill at least
%
\begin{equation}
 r\sin(\beta_2\vartheta_0/2) \geq \frac{1}{\cos(\beta_1\psi_0/2)}
\end{equation}
%
\NOTE{In fact, it can, but yields identically zero where the integration domain contains no zero of the delta argument.}%
since otherwise, the $\vartheta$ integral cannot be reduced with the $\delta$ distribution. 
The simultaneous conditions on $\psi_0$ and $\vartheta_0$ read as
%
\NOTE{Check if this makes sense. Geometrically, the condition on $\vartheta_0$ should look similar to the one for $\psi_0$.}%
\begin{align}
 r \sin(\beta_1\psi_0/2) &\geq 1 \\
 \sin(\beta_2\vartheta_0/2) &\geq \frac{1}{\sqrt{r^2 - 1}}.
\end{align}
%
Then, the integral over $\vartheta$ can be evaluated using
%
\begin{equation*}
 \delta\big(w(u)\big) = \sum_{u:w(u)=0} \frac{\delta(u)}{\ABS{w'(u)}}.
\end{equation*}
%
If we insert $\DUALOPU g$ into $\DUALOPP$ with the abbreviation $\Phi(\vartheta)$ for the integrand except the delta distribution, we get 
the $\vartheta$ integral
%
\begin{equation}
 \label{eq:conecurved:theta_integral_delta}
 \int_{-\beta_2\vartheta/2}^{\beta_2\vartheta/2} \Phi(\vartheta)\, 
 \delta\left( -\sin\vartheta \INNER{x}{\theta(\phi)} + \cos\vartheta \INNER{x}{e_z} - \sqrt{r^2 - \INNER{x}{\theta'(\phi)}^2} \sin\vartheta 
 \right)\, \D\vartheta.
\end{equation}
%
The zero of the argument $w(\vartheta)$ of $\delta$ can be calculated as
%
\begin{align}
 &\cos\vartheta \INNER{x}{e_z} - \sin\vartheta \Big( \INNER{x}{\theta(\phi)} + \sqrt{r^2 - \INNER{x}{\theta'(\phi)}^2} \Big) = 0 \quad
 \Rightarrow \notag \\
 \label{eq:conecurved:theta_in_delta}
 & \vartheta^* = \vartheta^*(\phi, x) = \arctan \left( \frac{\INNER{x}{e_z}}{\INNER{x}{\theta(\phi)} + \sqrt{r^2 - 
 \INNER{x}{\theta'(\phi)}^2}} \right),
\end{align}
%
and its derivative evaluated at this point is
%
\begin{equation*}
 w'(\vartheta^*) = \left. -\sin\vartheta \INNER{x}{e_z} - \cos\vartheta \Big( \INNER{x}{\theta(\phi)} + \sqrt{r^2 - 
 \INNER{x}{\theta'(\phi)}^2} \Big) \right|_{\vartheta = \vartheta^*}.
\end{equation*}
%
Using the relations $\sin\vartheta = \tan\vartheta / \sqrt{\tan^2\vartheta + 1}$ and $\cos\vartheta = 1 / \sqrt{\tan^2\vartheta + 1}$ and 
the abbreviations $a$ and $b$ for nominator and denumerator of the argument of $\arctan$ in \eqref{eq:conecurved:theta_in_delta}, we get 
\NOTE{Check if signs of $a$ and $b$ matter}%
%
\begin{equation*}
 \sin\vartheta^* = \frac{a}{\sqrt{a^2 + b^2}}, \quad \cos\vartheta^* = \frac{b}{\sqrt{a^2 + b^2}},
\end{equation*}
%
and in consequence
%
\begin{equation*}
 a\sin\vartheta^* + b\cos\vartheta^* = \frac{a^2}{\sqrt{a^2 + b^2}} + \frac{b^2}{\sqrt{a^2 + b^2}} = \sqrt{a^2 + b^2}.
\end{equation*}
%
The $\vartheta$ integral \eqref{eq:conecurved:theta_integral_delta} can thus be resolved to
%
\begin{equation*}
 \Phi\big(\arctan(a/b)\big) / \sqrt{a^2 + b^2}.
\end{equation*}
%
In total, we get the representation
%
\begin{align}
 \label{eq:conecurved:backproj}
 \DUALOPD g(x)
 &= \int_0^{2\pi} \frac{4}{\beta_1\beta_2\sqrt{r^2 - \sigma^2} \sqrt{a^2 + b^2}} \cdot \phantom{x} \\
 &\hspace{35pt} g_0 \big( \phi - \arcsin(\sigma/r), 2\arcsin(\sigma/r)/\beta_1, 2\arctan(a/b)/\beta_2 \big)\, \D\phi \notag
\end{align}
%
with
%
\begin{align}
 \label{eq:conecurved:bp_sigma}
 \sigma &= \sigma(\phi, x) = \INNER{x}{\theta'(\phi)}, \\
 \label{eq:conecurved:bp_a}
 a &= a(x) = \INNER{x}{e_z}, \\
 \label{eq:conecurved:bp_b}
 b &= b(\phi, x) = \INNER{x}{\theta(\phi)} + \sqrt{r^2 - \sigma(\phi,x)^2}.
\end{align}
\vspace{5ex}%




\subsection{Cone beam geometry -- helical acquisition}
\label{sec:applications:cone_helical}

In the helical or spiral geometry, both source and detector move on a helix instead of a circle. We assume the spiral to have constant 
pitch $\tilde p>0$, i.e. a full rotation by $2\pi$ corresponds to a $z$ shift by $\tilde p$. For simpler notation, we write 
$p = \tilde p/(2\pi)$. The source-detector system now executes $k > 0$ instead of one rotation in order to allow for objects which 
are extended in $z$ direction. We account for this situation by setting the domain of the preimage space functions as
%
\begin{equation}
 \label{eq:hcone:cylinderdomain}
 \Omega = B_1 \times [H_1, H_2]
\end{equation}
%
with $B_1 \subset \RR^2$ and $0 < H_1 < H_2$.

\subsubsection{Two-fold curved detector}
\label{sec:applications:cone_helical:twofoldcurved}

We now have the parameter sets
%
\begin{equation}
 \label{eq:hconecurved:params}
 T = [0, 2\pi k], \quad U = [-\psi_0, \psi_0] \times [-\vartheta_0, \vartheta_0].
\end{equation}
%
The detector parametrization can be derived from the circular acquisition situation \eqref{eq:conecurved:detector_parametr} by adding the 
shift $p \phi e_z$:
%
\begin{align}
 \label{eq:hconecurved:detector_parametr}
 X(\phi, \psi, \vartheta)
 &= \big[ r - 2R_2 \sin^2(\vartheta/2) - 2R_1 \cos\vartheta \sin^2(\psi/2) \big]\, \theta(\phi) + \big[-R_1 \sin\psi\big]\, 
 \theta'(\phi) + \phantom{x} \\
 &\hspace{14pt} \big[ -2R_1 \sin\vartheta\sin^2(\psi/2) + R_2\sin\vartheta + p\phi \big]\, e_z \notag 
\end{align}
%
Since both source and detector are shifted simultaneously, the direction field is the same as before,
%
\begin{equation}
 \label{eq:hconecurved:direction_field}
 N(\phi, \psi, \vartheta) = -\omega(\phi + \beta_1\psi/2, \beta_2\vartheta/2),
\end{equation} 
%
but the definition of the forward projection operator changes to
%
\begin{align}
 &\OPD: L^2(\Omega) \Lto L^2(T \times U) \notag \\
 \label{eq:hconecurved:fwdproj}
 &\OPD f(\phi, \psi, \vartheta) = \int_0^\infty f\big(-r\theta(\phi) + p\phi e_z + s \omega(\phi + \beta_1\psi/2, \beta_2\vartheta/2)\big)
 \D s.
\end{align}
%
To relate this operator to the X-ray transform, we again need to calculate the projection of the source point location onto $\omega$. Apart 
from the projection of $\theta(\phi)$ as calculated in section \ref{sec:applications:cone_circular:twofoldcurved}, we also need the 
projection of $e_z$. Using \eqref{eq:conecurved:omega1perp}--\eqref{eq:conecurved:omega2perp}, we see that the unit vector 
$\omega_1^\perp$ gives no contribution, and on the other hand,
%
\begin{equation*}
 \INNER{e_z}{\omega_2^\perp(\phi, \vartheta)} = \cos\vartheta.
\end{equation*}
%
\NOTE{TODO: formulate conditions on $l$ and $h$}%
It can be immediately concluded that
%
\begin{align*}
 \OPD f(\phi, \psi, \vartheta) 
 &= \OPP\big( \phi + \beta_1\psi/2, \beta_2\vartheta/2, r\sin(\beta_1\psi/2), r\cos(\beta_1\psi/2)\sin(\beta_2\vartheta/2) + 
 p\phi \cos(\beta_2\vartheta/2) \big) \\
 &= \OPU \OPP f(\phi, \psi, \vartheta).
\end{align*}
%
This operator $\OPU$ can be treated in the same way as the transform \eqref{eq:conecurved:coord_op}:
%
\begin{align*}
 \INNER{\OPU g}{h}
 &= \int_0^{\phi_0} \int_{-\psi_0}^{\psi_0} \int_{-\vartheta_0}^{\vartheta_0} g\big( \phi + \beta_1\psi/2, \beta_2\vartheta/2, 
 r\sin(\beta_1\psi/2), r \cos(\beta_1\psi/2) \sin(\beta_2\vartheta/2) + \phantom{x} \\
 &\hspace{80pt} p \phi \cos(\beta_2\vartheta/2) \big)\, h(\phi, \psi, \vartheta)\, \D\vartheta\, \D\psi\, \D\phi \\
 &= \int_0^{2\pi} \int_{-\psi_0}^{\psi_0} \int_{-\beta_2\vartheta_0/2}^{\beta_2\vartheta_0/2} g\big( \phi, \vartheta, 
 r\sin(\beta_1\psi/2), r \cos(\beta_1\psi/2) \sin\vartheta + (\phi - \beta_1\psi/2)\, p \cos\vartheta \big) \cdot \phantom{x} \\
 &\hspace{95pt} h_0\big( \phi - \beta_1\psi/2, \psi, 2\vartheta/\beta_2 \big)\, 
 \frac{2}{\beta_2}\, \D\vartheta\, \D\psi\, \D\phi \\
 &= \int_0^{2\pi} \int_{-r\sin(\beta_1\psi_0/2)}^{r\sin(\beta_1\psi_0/2)} \int_{-\beta_2\vartheta_0/2}^{\beta_2\vartheta_0/2}
 g\left( \phi, \vartheta, \sigma, \sqrt{r^2 - \sigma^2} \sin\vartheta + \big(\phi - \arcsin(\sigma/r)\big)\, p \cos\vartheta \right) \cdot 
 \phantom{x} \\
 &\hspace{135pt} h_0\big( \phi - \arcsin(\sigma/r), 2\arcsin(\sigma/r)/\beta_1, 2\vartheta/\beta_2 \big)\, \cdot \phantom{x} \\
 &\hspace{135pt} \frac{4}{\beta_1\beta_2\sqrt{r^2 - \sigma^2}}\, \D\vartheta\, \D\psi\, \D\phi \\
 &= \int_0^{2\pi} \int_{-r\sin(\beta_1\psi_0/2)}^{r\sin(\beta_1\psi_0/2)} \int_{-\beta_2\vartheta_0/2}^{\beta_2\vartheta_0/2} \int_{-r}^r
 g(\phi, \vartheta, \sigma, \tau) \cdot \phantom{x} \\
 &\hspace{145pt} \delta\left( \tau - \sqrt{r^2 - \sigma^2} \sin\vartheta - \big(\phi - \arcsin(\sigma/r)\big)\, p \cos\vartheta \right) 
 \cdot \phantom{x} \\
 &\hspace{145pt} h_0\big( \phi - \arcsin(\sigma/r), 2\arcsin(\sigma/r)/\beta_1, 2\vartheta/\beta_2 \big) \cdot \phantom{x} \\
 &\hspace{145pt} \frac{4}{\beta_1\beta_2\sqrt{r^2 - \sigma^2}}\, \D\tau\, \D\vartheta\, \D\sigma\, \D\phi.
\end{align*}
%
The delta distribution is reduced with the $\vartheta$ integral as for the circular acquisition, with the same arguments $\sigma$ and 
$\tau$. For the $\delta$ argument, we have
%
\begin{align*}
 w(\vartheta) 
 &= -\sin\vartheta \INNER{x}{\theta(\phi)} + \cos\vartheta \INNER{x}{e_z} - \sqrt{r^2 - \INNER{x}{\theta'(\phi)}^2} \sin\vartheta - 
 \big(\phi - \arcsin(\INNER{x}{\theta'(\phi)}/r)\big)\, p \cos\vartheta \\
 &= -\sin\vartheta \left( \INNER{x}{\theta(\phi)} + \sqrt{r^2 - \INNER{x}{\theta'(\phi)}^2} \right) + 
 \cos\vartheta \left( \INNER{x}{e_z} - p\, \big(\phi - \arcsin(\INNER{x}{\theta'(\phi)}/r)\big) \right)
\end{align*}
%
which has its zero at
%
\begin{equation}
 \label{eq:hconecurved:theta_in_delta}
 \vartheta^*(\phi, x) = \arctan \left( \frac{\INNER{x}{e_z} - p\, \big(\phi - \arcsin(\INNER{x}{\theta'(\phi)}/r)}{\INNER{x}{\theta(\phi)} + 
 \sqrt{r^2 - \INNER{x}{\theta'(\phi)}^2}} \right)
\end{equation} 
%
or in short $\vartheta^* = \arctan\big((a - pd) / b\big)$ with $a, b$ as in \eqref{eq:conecurved:bp_a}--\eqref{eq:conecurved:bp_b} and
%
\begin{equation}
 \label{eq:hconecurved:bp_d}
 d = d(\phi, x) = \big(\phi - \arcsin(\INNER{x}{\theta'(\phi)}/r)\big).
\end{equation}
%
Inserting $\vartheta^*$ into $w'(\vartheta)$ yields 
%
\begin{align*}
 w'(\vartheta) 
 &= -b \cos(\vartheta^*) + (a - pd)\sin(\vartheta^*) \\
 &= -b \frac{1}{\sqrt{\tan^2\vartheta^* + 1}} - (a - pd) \frac{\tan\vartheta^*}{\sqrt{\tan^2\vartheta^* + 1}} \\
 &= -b \frac{b}{\sqrt{(a-pd)^2 + b^2}} - (a - pd) \frac{a - pd}{\sqrt{(a-pd)^2 + b^2}} \\
 &= - \sqrt{(a-pd)^2 + b^2}.
\end{align*}
%
\NOTE{TODO: work out conditions for the integration domain to contain a zero of $w$}%
Hence, the $\vartheta$ integral reduces to
%
\begin{align*}
 \int_{-\beta_1\vartheta_0/2}^{\beta_1\vartheta_0/2} \Phi(\vartheta)\, \delta \big(w(\vartheta)\big) \, \D\vartheta = 
 \frac{\Phi(\vartheta^*)}{\sqrt{(a-pd)^2 + b^2}},
\end{align*}
%
and thus we acquire the backprojection
%
\begin{align}
 \label{eq:hconecurved:backproj}
 \DUALOPD g(x)
 &= \int_0^{2\pi} \frac{4}{\beta_1\beta_2\sqrt{r^2 - \sigma^2} \sqrt{(a-pd)^2 + b^2}} \cdot \phantom{x} \\
 &\hspace{35pt} g_0 \left( \phi - \arcsin(\sigma/r), 2\arcsin(\sigma/r)/\beta_1, 2\arctan\big((a-pd)/b\big)/\beta_2 \right)\, \D\phi. \notag
\end{align}
\vspace{5ex}%



\subsubsection{Detector curved only along $\phi$}
\label{sec:applications:cone_helical:curvedstack}

Straightforward application of the arguments from section \ref{sec:applications:cone_circular:curvedstack}. TODO
\vspace{5ex}%


\subsubsection{Flat detector}
\label{sec:applications:cone_helical:flat}

Straightforward application of the arguments from section \ref{sec:applications:cone_circular:flat}. TODO
\vspace{5ex}%


\subsection{PET and SPECT geometries -- big TODO}


\end{document}
